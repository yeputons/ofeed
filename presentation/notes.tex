\documentclass[a4paper]{article}
\usepackage{cmap}
\usepackage[T2A]{fontenc}
\usepackage[utf8]{inputenc}
\usepackage[english,russian]{babel}
\usepackage{fancyhdr}
\usepackage{indentfirst}
\usepackage[top=2cm,bottom=2cm,left=1cm,right=1cm]{geometry}
\usepackage[page]{totalcount}
\pagestyle{fancy}

\renewcommand{\thesection}{}

\begin{document}

\lhead{Презентация проекта 21.03.2016}
\rhead{Егор Суворов}
\cfoot{Страница \thepage\ из \totalpages}

\section{Слайд 1}

Добрый день, меня зовут Егор Суворов, я представляю проект "Оффлайн просмотр ленты ВКонтакте"

\section{Слайд 2}

Предыстория проекта такова: я регулярно нахожусь в метро без сотовой связи и интернета; а
заниматься чем-нибудь по-прежнему хочется.
Разложить ноутбук или тетрадку в час пик невозможно, поэтому продуктивно провести время не получается.
Обычно в таком случае я смотрю в ленту ВК, но официальные клиенты плохо работают с кэшированием.
Например, если я загрузил последние посты, а потом через час попробовал загрузить следующие, то с большой
вероятностью приложение сбросит кэш из предыдущих постов, так как они были слишком давно.
Также официальные клиенты никак не работают со ссылками, приложенными к постам - если у меня нет интернета,
я никак не могу открыть даже Вики-страницу внутри ВК, не говоря уже о статьях на внешних сайтах.

\section{Слайд 3}

Поставленная мной задача "--- реализовать клиент просмотра ленты ВК, работающий по идеям offline-first.
Это означает, что отсутствие связи является нормой для приложение, в отличие от стандартного подхода,
где отсутствующая связь - исключительная ситуация.
Клиент должен агрессивно кэшировать все данные: единожды загрузив что-либо, он не должен удалять это из кэша
без явной команды от пользователя.
В том числе клиент должен уметь скачивать страницы, ссылки на которые прикреплены к посту, причём включая
таблицы стилей и картинки, чтобы для просмотра страницы интернет не требовался.

Однако такой подход несёт трудности: лента может получиться разрезанной на несколько независимых частей,
между которыми могут быть незагруженные посты.
Скажем, если я загрузил 50 постов вчера вечером и 50 постов только что, то с большой вероятностью я не подгрузил
посты, сделанные сегодняшним утром.
При этом терять вчерашние посты нельзя, более того "--- их надо корректно отобразить в ленте и дать пользователю
возможность догрузить посты между двумя кусками.

\section{Слайд 4}
Вот пример разрыва ленты.
Тут есть посты 103, 104 и 107, причём приложение не уверено, что между постами 104 и 107 нет каких-то других постов,
потому что они были загружены в разные моменты времени.
Именно поэтому оно отображает кнопку <<Load feed here>>, которую я могу нажать и приложение начнёт грузить это место ленты.

\section{Слайд 5}
Картинка получится такая.
После окончания загрузки я увижу следующее:

\section{Слайд 6}
Здесь приложение, скорее всего, получило в ответе от сервера посты 106, 105 и 104, узнав, таким образом,
что между постами 107 и 104 никаких других нет.
Теперь оно отобразило новые посты, а кнопку <<Load feed here>> убрало.

\section{Слайд 7}
На этом слайде представлены используемые библиотеки и те части, для которых теоретически уже могла бы быть написана библиотека.
Во-первых, я использую ORMLite для общения с базой данных, где лежит весь кэш приложения (кроме веб-страниц и картинок).
Во-вторых, я использую библиотеку JSoup для работы с некорректными HTML-страницами, а таких в интернете большинство.
Она позволяет не только находить используемые в исходном коде страницы картинки, но и изменять ссылки на них, подменяя исходные
ссылки на скачанные файлы.
В-третьих, я использовал библиотеку, предоставляемую В Контакте разработчикам под Android.
Она предоставляет удобные классы и методы для работы с объектами VK Api.
К сожалению, это не относится к методам для работы с лентой новостей "--- эту часть пришлось дописать самостоятельно.

Также я самостоятельно реализовал логику загрузки и обработки HTML-страниц для оффлайн-просмотра.
Мне показалось это довольно интересной задачей.

\section{Слайд 8}
Одним из самых сложных мест ожидаемо оказалась логика хранения и загрузки ленты, состоящей из нескольких кусков.
Дополнительной сложностью стала необходимость такой реализации, которая позволила бы отобразить ленту в ListView
с возможностью сохранения положения прокрутки при догрузке ленты.
Например, оказалось, что очень удобно хранить в базе данных не только информацию о загруженных постах,
но и специальные посты-заглушки в тех местах, где предполагается кнопка <<Load feed here>>.
Над их добавлением и удалением пришлось поработать.

Также оказалось, что реализация отображения ленты <<в лоб>> очень сильно тормозит из-за постоянного
создания объектов на экране и прокрутка получается медленной. Мне пришлось разобраться, как в Android принято решать эту проблему.
Самой неожиданной проблемой стала невозможность легко и просто открыть локально сохранённую HTML-страницу в браузере.
Далеко не первый найденный в гугле метод смог корректно загрузить файл с карты памяти, а не из интернета.

За бортом же осталась поддержка всех типов вложений "--- на данный момент поддерживаются только фотографии и ссылки.
Также не отображаются комментарии и лайки к постам.
Возможность детально управлять закэшированными данными также отсутствует "--- можно только сбросить вообще весь кэш.
Отсюда также следует не всегда корректная обработка изменённых постов "--- если пост изменился, нет никакого способа попросить
приложение перегрузить только его.

\section{Слайд 9}
С прошлой презентации в декабре функциональность приложения не изменилась, однако были сделаны изменения внутри кода.
Я подчистил некоторые несостыковки в стиле кода и добавил автоматические проверки при помощи плагина checkstyle.
Оказалось, что код, дописанный к VK SDK не всегда следует Java Coding Conventions, поэтому пришлось дописать для него
отдельный набор проверок и узнать, как менять правила в зависимости от проверяемого файла.

\section{Слайд 10}
Спасибо за внимание!
На этом слайде есть ссылки на исходники, скомпилированный apk-файл для установки и мой адрес.
Вопросы?

\end{document}
